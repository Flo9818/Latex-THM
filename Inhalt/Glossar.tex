% !TEX encoding = UTF-8 Unicode
% !TEX root =  ../Arbeit.tex

%
% Abkürzung --> referenz, kurz, lang
% Glossareintraege --> referenz, name, beschreibung
% Aufruf mit \gls{...}
%

%% -----
% Abkürzungen
%% -----


%% -----
% Einfache Akronyme
%% -----
\newacronym{B2B}{B2B}{Business-to-Business}
\newacronym{B2C}{B2C}{Business-to-Consumer}
\newacronym{CRM}{CRM}{Customer-Relation-Managment}
\newacronym{ERP}{ERP}{Enterprise-Ressource-Planning}
\newacronym{GUI}{GUI}{Graphical User Interface}
\newacronym{HTTP}{HTTP}{Hypertext Transfer Protocol}
\newacronym{IoT}{IoT}{Internet of Things}
\newacronym{UI}{UI}{User Interface}
\newacronym{WEB}{Web}{World Wide Web}


%% -----
% Akronyme die ins Glossar aufgenommen werden
%% -----
\newglossaryentry{apig}{name={API},
	description={Eine API ist eine Schnittstelle, die ein Softwaresystem bereitstellt, um dieses in andere Programme einzubinden}}

%%% define the acronym and use the see= option
\newglossaryentry{api}{type=\acronymtype, name={API}, description={Application
		Programming Interface}, first={Application
		Programming Interface (API)\glsadd{apig}}, see=[Glossar:]{apig}}

\newglossaryentry{hapi}{
	name={hapi.js},
	description={Ein Javascript-Plugin, welches ein dynamisches Erstellen eines Webservers mitsamt Routen und Eingabevalidierung erlaubt}
}
\newglossaryentry{RESTg}{
	name={REST},
	description={Ein Programmierparadigma für verteilte Systeme, insbesondere Webservices, welches eine Abstraktion des HTTP Architekturschemas darstellt}
}

\newglossaryentry{REST}{type=\acronymtype, name={REST}, description={Representational State Transfer}, first={Representational State Transfer (REST)\glsadd{RESTg}}, see=[Glossar:]{RESTg}}

\newglossaryentry{node}{
	name={node.js},
	description={Eine serverseitige Plattform zur Softwareentwicklung zum Betrieb von Netzwerkanwendungen}
}
\newglossaryentry{typescript}{
	name={Typescript},
	description={Eine von Microsoft entwickelte, auf Javascript basierende Programmiersprache, welche typische Sprachkonstrukte der objektorientierten Programmierung unterstützt}
}
